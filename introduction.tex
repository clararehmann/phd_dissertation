% introduction outline

Patterns of genetic diversity within and between populations can be leveraged to reveal their evolutionary history.

Explain the patterns - what do you look for, what does it mean, examples

In real populations, these patterns are also distributed across geographic space, with each individual in a population
 - and their genome - occupying a location on the landscape. 
As evolutionary time plays forward, these individuals interact, reproduce, and disperse locally, giving rise to spatial 
genetic structure within populations: in short, individuals that are closer spatially tend to also be closer genetically.
While this pattern, termed "isolation by distance" by Sewall Wright \comment{CITATION}, is ubiquitous in natural populations and 
can confound and evolutionary inference, understanding the influence of spatial structure on observed patterns of genetic diversity
also adds another dimension of information that can be leveraged to uncover the evolutionary processes acting on organisms.

\comment{This dissertation focuses on leveraging spatial population structure to uncover and describe evolutionary processes 
in the malaria parasite, \textit{Plasmodium}, and its mosquito vector, \textit{Anopheles}.}
% this is just gonna be bad writing that i'll workshop
Despite recent advances in treatments, vaccines, and XYZ, malaria remains a persistent public health threat, particularly in sub-Saharan Africa.
Understanding the evolutionary histories of these species is crucial for malaria control efforts: both the parasite and vector
in this system are rapidly evolving to counter human efforts to curb the spread of disease.
Antimalarial resistance has been reported in \comment{XYZ strains, N number of times...}, and \textit{Anopheles} species readily
evolve resistance to insecticides deployed for vector control.
Vector control efforts in particular are critical for containing the spread of malaria, and extensive genomic surveillance
of \textit{Anopheles} species can now reveal \comment{when and how often these mosquitoes are adapting to human interventions}.
In these organisms, \comment{incorporating spatial information} into our understanding of their genetic diversity and evolutionary histories
adds another layer of data that \comment{will help us uncover processes that go undetected using standard population genetic methods}.

